\talk
{%
    Gaofeng Huang
}
{%
    Large holomorphic automorphism groups
}
{%
    In this talk, we survey a few results in the study of large
    holomorphic automorphism groups. First we give an historical
    account of the so-called Andersen-Lempert theory, the core of
    which is an approximation of local biholomorphic injections by
    global holomorphic automorphisms, developed by Andersen-Lempert
    and Forstneric-Rosay in the 90s. Such an approximation is
    possible on Stein manifolds with the density property, a
    property describing the abundance of globally integrable
    holomorphic vector fields.  It thus is fundamental to identify
    Stein manifolds with this property. A criterion by Kaliman and
    Kutzschebauch has substantially enlarged the classes of
    examples. We will also encounter two recent developments, one is
    a generalization of this criterion, and the other is a
    specification of this criterion to smooth affine $SL_2$-varieties.
}
