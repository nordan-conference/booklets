\talk{Gianmarco Brocchi}{Progress on the Kate square root estimate}
{%
    The Kato square root estimate is a $L^2$ inequality concerning
    perturbations of the Laplacian. While the one-dimensional case was
    established in the 1980s by Coifman, McIntosh, and Meyer,
    the higher-dimensional extension --- where the perturbation takes the
    form of a matrix-valued function $A$ in the divergence-form operator
    $-\mathrm{div}(A \nabla)$ --- remained open for two more decades.

    In this talk, I will introduce the Kato square root estimate and
    describe the \emph{first-order method}, a technique that reduces the
    second-order operator $-\mathrm{div}(A \nabla)$ to a first-order,
    bisectorial operator $DB$. This method exploits a connection between
    harmonic and holomorphic extensions and allows us to rewrite the
    original estimate as a question about the boundedness of the
    holomorphic functional calculus for $D B$.

    I will also present recent results in the theory, including extensions
    to Riemannian manifolds and to operators with \emph{degenerate
    coefficients}, where the matrix $A(x)$ may lack uniform bounds or
    accretivity and can exhibit singular behaviour. What types of
    singularities can be handled? On which classes of manifolds? And in
    Euclidean space, can one treat anisotropic singularities, namely
    those that vary with direction? 

    New results are part of ongoing joint work with Andreas Rosén.  
}
