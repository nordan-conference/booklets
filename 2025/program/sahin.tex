\talk{Sibel Şahin}{Approximation Numbers: From Kolmogorov Numbers\\to Differences of Composition Operators}
{
    Joint work with Frédéric Bayart of Laboratoire de Mathématiques Blaise Pascal.

    In this talk we will first consider various singular numbers of operators
    which happen to be equivalent in the Hilbert space setting. Through
    Kolmogorov numbers we will first see how these singular entities for
    composition operators relate to complex potential theory, namely
    Monge-Amp`ere capacity. In the second part we will relate the component
    structure of bounded composition operators to the function theoretic
    properties of the symbols and for this we will focus on the approximation
    numbers of differences of composition operators. We will see how one can
    obtain optimal upper and lower bounds for approximation numbers of
    differences using classical singular invariants like Bernstein and
    Gelfand numbers and specific choices of Blaschke products from the
    underlying function space.

    \bigskip
    \noindent
    \textit{\textbf{\large Literature}}

    \medskip
    \textsc{G. Lechner, D. Li, H. Queffélec, L. Rodriguez-Piazza}
    
    \hfill \textit{Approximation numbers of weighted composition operators}

    \hfill Journal of Functional Analysis 274, 1928–1958 (2018)

    \textsc{J. Moorhouse, C. Toews}
    \hfill \textit{Differences of composition operators}

    \hfill Contemporary Mathematics 321, 207–213 (2003)

    \textsc{H. Queffélec, K. Seip}
    \hfill \textit{Decay rates for approximation numbers}

    \hfill \textit{of composition operators}

    \hfill Journal d’Analyse Mathématique 125, 371–399 (2015)%
}
