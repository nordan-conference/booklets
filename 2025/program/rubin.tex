\talk{Olof Rubin}{Chebyshev polynomials on equipotential curves}
{%
    Given a compact set $K\subset \mathbb C$, a Chebyshev polynomial
    is a monic polynomial that minimizes the supremum norm on $K$.
    When $K$ is infinite such a polynomial exists uniquely for each
    degree. Although there are no explicit formulas for computing
    Chebyshev polynomials, they can be studied through families of
    near-minimal polynomials. One such family is that of Faber
    polynomials, which arise naturally from the conformal map of the
    complement of $K$ onto the exterior of the unit disk. In this
    talk, I will present recent results establishing connections
    between Chebyshev and Faber polynomials on equipotential curves.
}
